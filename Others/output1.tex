
    \documentclass{article}
    \begin{document}
     \newline
 \newline
 \newline
 \newline
 \newline
A-level \newline
 \newline
ACCOUNTING \newline
 \newline
7127/1 \newline
 \newline
Paper 1  Financial Accounting \newline
 \newline
Mark scheme \newline
 \newline
June 2022 \newline
 \newline
Version: 1.0 Final Mark Scheme \newline
 \newline
*226A7127/1/MS* \newline
MARK SCHEME – A-LEVEL ACCOUNTING – 7127/1 – JUNE 2022 \newline
2 \newline
Mark schemes are prepared by the Lead Assessment Writer and considered, together with the relevant \newline
questions, by a panel of subject teachers.  This mark scheme includes any amendments made at the \newline
standardisation events which all associates participate in and is the scheme which was used by them in \newline
this examination.  The standardisation process ensures that the mark scheme covers the students’ \newline
responses to questions and that every associate understands and applies it in the same correct way.  \newline
As preparation for standardisation each associate analyses a number of students’ scripts.  Alternative \newline
answers not already covered by the mark scheme are discussed and legislated for.  If, after the \newline
standardisation process, associates encounter unusual answers which have not been raised they are \newline
required to refer these to the Lead Examiner. \newline
 \newline
It must be stressed that a mark scheme is a working document, in many cases further developed and \newline
expanded on the basis of students’ reactions to a particular paper.  Assumptions about future mark \newline
schemes on the basis of one year’s document should be avoided; whilst the guiding principles of \newline
assessment remain constant, details will change, depending on the content of a particular examination \newline
paper. \newline
 \newline
 \newline
Further copies of this mark scheme are available from aqa.org.uk \newline
 \newline
   \newline
 \newline
Copyright information  \newline
 \newline
AQA retains the copyright on all its publications.  However, registered schools/colleges for AQA are permitted to copy material from this booklet for their own \newline
internal use, with the following important exception: AQA cannot give permission to schools/colleges to photocopy any material that is acknowledged to a third \newline
party even for internal use within the centre.  \newline
 \newline
Copyright © 2022 AQA and its licensors.  All rights reserved.  \newline
MARK SCHEME – A-LEVEL ACCOUNTING – 7127/1 – JUNE 2022 \newline
3 \newline
Level of response marking instructions \newline
 \newline
Level of response mark schemes are broken down into levels, each of which has a descriptor.  The \newline
descriptor for the level shows the average performance for the level.  There are marks in each level. \newline
 \newline
Before you apply the mark scheme to a student’s answer read through the answer and annotate it (as \newline
instructed) to show the qualities that are being looked for.  You can then apply the mark scheme. \newline
 \newline
Step 1 Determine a level \newline
 \newline
Start at the lowest level of the mark scheme and use it as a ladder to see whether the answer meets the \newline
descriptor for that level.  The descriptor for the level indicates the different qualities that might be seen in \newline
the student’s answer for that level.  If it meets the lowest level then go to the next one and decide if it \newline
meets this level, and so on, until you have a match between the level descriptor and the answer.  With \newline
practice and familiarity you will find that for better answers you will be able to quickly skip through the \newline
lower levels of the mark scheme. \newline
 \newline
When assigning a level you should look at the overall quality of the answer and not look to pick holes in \newline
small and specific parts of the answer where the student has not performed quite as well as the rest.  If \newline
the answer covers different aspects of different levels of the mark scheme you should use a best fit \newline
approach for defining the level and then use the variability of the response to help decide the mark within \newline
the level, ie if the response is predominantly level 3 with a small amount of level 4 material it would be \newline
placed in level 3 but be awarded a mark near the top of the level because of the level 4 content. \newline
 \newline
Step 2 Determine a mark \newline
 \newline
Once you have assigned a level you need to decide on the mark.  The descriptors on how to allocate \newline
marks can help with this.  The exemplar materials used during standardisation will help.  There will be an \newline
answer in the standardising materials which will correspond with each level of the mark scheme.  This \newline
answer will have been awarded a mark by the Lead Examiner.  You can compare the student’s answer \newline
with the example to determine if it is the same standard, better or worse than the example.  You can then \newline
use this to allocate a mark for the answer based on the Lead Examiner’s mark on the example. \newline
 \newline
You may well need to read back through the answer as you apply the mark scheme to clarify points and \newline
assure yourself that the level and the mark are appropriate. \newline
 \newline
Indicative content in the mark scheme is provided as a guide for examiners.  It is not intended to be \newline
exhaustive and you must credit other valid points.  Students do not have to cover all of the points \newline
mentioned in the Indicative content to reach the highest level of the mark scheme. \newline
 \newline
An answer which contains nothing of relevance to the question must be awarded no marks. \newline
 \newline
 \newline
 \newline
MARK SCHEME – A-LEVEL ACCOUNTING – 7127/1 – JUNE 2022 \newline
4 \newline
Marking guidance question 11 \newline
 \newline
This question is only testing Assessment Objective 1. \newline
 \newline
You should apply the level of response mark scheme to each advantage.  \newline
 \newline
Read the advantage as a whole and decide if it is clear, partial, fragmented or nothing worthy of credit. \newline
 \newline
When you have made your decision, award the appropriate level by using L3, L2, L1 or L0 and show this \newline
on the answer by using the drop-down comment box.   \newline
 \newline
Then put the appropriate mark for the question in the mark box. \newline
 \newline
 \newline
 \newline
Marking guidance for questions 14.2 \& 15.2 \newline
 \newline
These questions are testing Assessment Objectives 2 and 3. \newline
 \newline
Be clear on the focus of the question. \newline
 \newline
Read the whole question and decide which level should be awarded, then add the appropriate level to \newline
the script from the comment box, eg L3, L2, L1 or L0. \newline
 \newline
Then put the marks awarded for the question in the mark box. \newline
 \newline
Remember that the indicative content provides possible answers but there may be others that are \newline
equally valid, and you should give credit to other lines of argument. \newline
 \newline
A good response does not need to include all the indicative content. \newline
 \newline
 \newline
 \newline
 \newline
 \newline
 \newline
 \newline
 \newline
 \newline
 \newline
MARK SCHEME – A-LEVEL ACCOUNTING – 7127/1 – JUNE 2022 \newline
5 \newline
Marking guidance for questions 16 \& 17 \newline
 \newline
Be clear on the focus of the question. \newline
 \newline
When you have decided on the level to be awarded add the appropriate comment which best describes \newline
the response to the end of the answer. \newline
 \newline
L5 \newline
Convincing judgement/recommendation fully supported by evaluation and analysis \newline
of a wide range of evidence with a strong chain of reasoning. \newline
 \newline
L5 \newline
Astute judgement/recommendation which takes limitations of evidence into account. \newline
 \newline
L4 \newline
Judgement/recommendation is supported by evaluation and analysis of a range of \newline
evidence. \newline
 \newline
L4 \newline
Judgement/recommendation is supported after some consideration of limitations of \newline
evidence. \newline
 \newline
L3 \newline
Judgement/recommendation is incomplete but supported by analysis of a range of \newline
evidence. \newline
 \newline
L3 \newline
Judgement/recommendation is developed but analysis and application are limited. \newline
 \newline
L2 \newline
Judgement/recommendation is given but with limited analysis and weak application. \newline
 \newline
L2 \newline
Limited range of evidence is analysed and limited chain of reasoning. \newline
 \newline
L1 \newline
Fragmented points of little/unclear relevance. \newline
 \newline
L1 \newline
No conclusion/recommendation/judgement with poor application. \newline
 \newline
L1 \newline
Conclusion unsupported. \newline
 \newline
L0 \newline
Has not attempted the question. \newline
 \newline
L0 \newline
Has not produced an answer of any value. \newline
 \newline
 \newline
Then review the script and annotate using the following comments:  \newline
 \newline
Where you identify: \newline
Situation \newline
Comment to use \newline
Application \newline
knowledge of \newline
principles/concepts/techniques \newline
Application is fragmented or \newline
descriptive or not adequately \newline
applied to the context \newline
Weak application \newline
Application is relevant and \newline
applied fully to the context \newline
Clear application \newline
Analysis  \newline
A limited attempt at analysis \newline
Weak analysis \newline
Analysis is logical/considered \newline
Reasoned analysis \newline
MARK SCHEME – A-LEVEL ACCOUNTING – 7127/1 – JUNE 2022 \newline
6 \newline
Evaluation \newline
An attempt at \newline
assessment/evaluation with \newline
little or no supporting \newline
evidence \newline
Weak evaluation \newline
Evaluation/assessment is \newline
logical and supported by \newline
evidence \newline
Supported evaluation \newline
Evaluation/assessment \newline
considers the relative \newline
significance and limitations of \newline
the evidence \newline
Astute evaluation \newline
Judgement or Conclusion or \newline
Recommendation  \newline
An attempt at judgement is \newline
made but unsupported by \newline
evidence or argument \newline
Judgement/conclusion – \newline
unsupported  \newline
A judgement is made and is \newline
supported but the support is \newline
weak or evidence used is \newline
limited \newline
Judgement/conclusion – \newline
limited/weak support \newline
The judgement is supported \newline
by evidence and argument \newline
but may not be fully balanced \newline
Judgement/conclusion –\newline
supported  \newline
Judgement is supported by \newline
evidence and considers the \newline
limitations of the evidence in \newline
context \newline
Judgement/conclusion – \newline
fully justified  \newline
 \newline
Remember that the indicative content provides possible answers but there may be others that are \newline
equally valid and you should give credit to other lines of argument.  \newline
 \newline
A good response does not need to include all the indicative content.  \newline
 \newline
Consider the question as a whole, together with the annotations made, and decide on the level to be \newline
awarded. \newline
 \newline
Show the Level awarded, eg L2 using the relevant comment from the drop-down list and then enter the \newline
mark in the total box reflecting where in the level the answer sits. \newline
 \newline
If in doubt about an answer or if you are unsure of the validity of the content then contact your Team \newline
Leader.  Please make sure that you follow the guidance in the standardisation scripts as we need to \newline
have a consistent approach across all marking. \newline
 \newline
Be positive in your marking and look to reward what is there. \newline
 \newline
 \newline
MARK SCHEME – A-LEVEL ACCOUNTING – 7127/1 – JUNE 2022 \newline
7 \newline
The own figure rule  \newline
 \newline
General principle  \newline
 \newline
The own figure rule is designed to ensure that students are only penalised once for a particular error at \newline
the point at which that error is made, and suffer no further penalty as a consequence of the error.  The \newline
error could be in an account, a calculation, financial statement, or prose explanation.  Where the own \newline
figure rule is to be applied in a mark scheme, the symbol OF is used.  \newline
 \newline
Applications  \newline
 \newline
In an account: a student could still achieve a mark for balancing an account with their own figure, rather \newline
than the correct figure, if they had made an error in the account (such as the omission of an entry, or the \newline
inclusion of an incorrect figure for an otherwise valid entry).  However, it should be noted that an own \newline
figure would not be awarded for the balance of an account, if the account contained any item which \newline
should not have appeared (often referred to as an ‘alien’ item).  \newline
 \newline
In a complex calculation to which several marks are allocated: a student could achieve an own figure \newline
mark for the result of a complex calculation, if an error has been made in one of the steps leading to the \newline
final result.  The complex calculation could be a separate task, or an aspect of a larger requirement \newline
(such as workings to provide details for a financial statement).  \newline
 \newline
In a financial statement: a student could still achieve a mark for calculating an own figure for a key \newline
subtotal within a financial statement where an error had already occurred in the data making up the \newline
subsection (such as the omission of an item, or an incorrect figure for an otherwise valid entry).  Again, \newline
the own figure for a subtotal would not be given if the subsection included any ‘alien’ item.  \newline
 \newline
In a prose statement: a student who is explaining or interpreting some financial statements or data that \newline
they have prepared but which contains errors, would be credited with an appropriate interpretation of \newline
their own figures.  \newline
 \newline
Workings  \newline
 \newline
A ‘W’ next to a figure in the mark schemes means that the figure needs to be calculated by the student to \newline
which workings are shown for reference.  If the figure the student has given in their answer is wrong and \newline
the marks given for that calculation are more than 1 then the marker must refer to the working for that \newline
item.  The working will show the steps of the calculation to which the marks are attributed and the \newline
student should be allocated the marks for the steps they completed correctly. \newline
 \newline
Financial Statements \newline
 \newline
Where questions require students to prepare financial statements (or extracts of these), the indicative \newline
mark schemes will include the standard wording that we have suggested in our statements resource. \newline
Students may use slightly different wording for entries in statements, which should not be penalised if it \newline
is clear that the meaning is the same. \newline
 \newline
 \newline
 \newline
MARK SCHEME – A-LEVEL ACCOUNTING – 7127/1 – JUNE 2022 \newline
8 \newline
Section A \newline
 \newline
Multiple Choice Questions \newline
 \newline
Question \newline
Number \newline
Answer \newline
01 \newline
D \newline
Neither statement is true. \newline
02 \newline
A \newline
Income and expenditure are matched to the period they belong to. \newline
03 \newline
D \newline
Current assets − closing inventory\newline
 Current liabilities \newline
 \newline
04 \newline
C \newline
Interest on drawings, partner salaries, share of residual profits \newline
05 \newline
C \newline
£16 900 \newline
06 \newline
C \newline
£1 879 \newline
07 \newline
B \newline
£525 Dr \newline
08 \newline
C \newline
£37 800 \newline
09 \newline
C \newline
£245 \newline
10 \newline
B \newline
Partial omission \newline
 \newline
[1 mark for each correct answer] \newline
 \newline
 \newline
MARK SCHEME – A-LEVEL ACCOUNTING – 7127/1 – JUNE 2022 \newline
9 \newline
Qu \newline
Part \newline
Marking Guidance \newline
Total \newline
marks \newline
11 \newline
 \newline
Explain two advantages of using a bank overdraft as a source of finance for a \newline
business. \newline
6 \newline
 \newline
AO1 – 6 marks \newline
 \newline
Apply the levels of response mark scheme to each advantage – maximum three marks for each \newline
advantage. \newline
 \newline
Level \newline
Marks \newline
Description \newline
3 \newline
3 \newline
A clear and thorough explanation showing understanding of an advantage of using a \newline
bank overdraft as a source of finance. \newline
2 \newline
2 \newline
A partial explanation showing understanding, but lacking detail and/or minor \newline
inaccuracies. \newline
1 \newline
1 \newline
Fragmented points made. \newline
0 \newline
0 \newline
Nothing worthy of credit. \newline
 \newline
Answers may include: \newline
 \newline
• Flexibility – can change the amount borrowed at any time as long as the business are within the \newline
overdraft limit set by the bank. \newline
• Cost – interest is only paid on amounts borrowed so it is cost effective in that if the business pay it off \newline
quickly it will reduce the overall cost and there is not normally a charge for early payment. \newline
• Speed – overdrafts can usually be arranged / amended quickly so that the business can access the \newline
money when needed to meet short-term obligations eg wages, trade payables. \newline
• Risk – overdrafts are not usually secured against non- current assets so the business should not be at \newline
risk of losing their non-current assets. \newline
 \newline
Marker note: \newline
 \newline
Do not reward disadvantages. \newline
 \newline
Not all content needs to be covered to gain full marks. \newline
 \newline
The indicative content is not exhaustive other credit worthy material should be awarded marks as \newline
appropriate. \newline
 \newline
 \newline
 \newline
MARK SCHEME – A-LEVEL ACCOUNTING – 7127/1 – JUNE 2022 \newline
10 \newline
Qu \newline
Part \newline
Marking Guidance \newline
Total \newline
marks \newline
12 \newline
 \newline
Prepare the rent receivable account for the year ended 31 March 2022.   \newline
Show clearly the amount to be transferred to the income statement and the \newline
balance brought down 1 April 2022.  \newline
5 \newline
 \newline
AO1 – 5 marks \newline
 \newline
Dr \newline
Rent receivable \newline
Cr \newline
Details \newline
Amount \newline
 £ \newline
Details \newline
Amount \newline
 £ \newline
Balance b/d \newline
860 (1) \newline
Bank \newline
13 900 * \newline
Bank \newline
500 (1)* both  \newline
    \newline
Income Statement \newline
10 720 (1)** OF \newline
 \newline
    \newline
Balance c/d \newline
1 820 *** \newline
 \newline
    \newline
 \newline
13 900 \newline
   \newline
13 900   \newline
 \newline
  \newline
  \newline
Balance b/d \newline
1 820 (2) W1 *** \newline
 \newline
W1  \newline
£2 730 / 3 × 2 = £1 820 (1) \newline
 \newline
Marker note: \newline
 \newline
* Award 1 mark for both entries.  Do not award where the student has netted the figures off ie £13 400 \newline
** Award 1 mark for the income statement figure being used to balance the account – this could be on \newline
the debit or the credit side.  This must be arithmetically correct. \newline
*** Award 1 mark for the calculation of £1 820.  Award 1 further mark for the balance brought down.  This \newline
must be on the credit side of the account and must use one of the following figures (£1 820, £910 or £2 \newline
730) to be awarded. \newline
 \newline
To award the marks amounts must be supported by an appropriate label (recognisable abbreviations are \newline
acceptable for Balance, eg Bal c/d, b/d, Bal c/fwd., b/fwd. For the income statement label accept profit \newline
and loss account, I+S and P+L. For the bank label accept refund.  \newline
 \newline
 \newline
 \newline
MARK SCHEME – A-LEVEL ACCOUNTING – 7127/1 – JUNE 2022 \newline
11 \newline
Qu \newline
Part \newline
Marking Guidance \newline
Total \newline
marks \newline
13 \newline
1 \newline
Prepare Ross’s suspense account to correct the errors, clearly showing the \newline
opening balance. \newline
5 \newline
 \newline
AO1 – 5 marks \newline
 \newline
Dr \newline
Suspense \newline
 \newline
Cr \newline
Details \newline
Amount \newline
£ \newline
Details \newline
Amount \newline
£ \newline
Balance b/d \newline
19 800 (1) OF * Rent payable \newline
15 200 (1) \newline
Discount received \newline
2 400 (1) \newline
Wages \newline
3 500 (1) \newline
  \newline
    \newline
Drawings \newline
3 500 (1) \newline
  \newline
22 200  \newline
  \newline
22 200  \newline
 \newline
Marker note: \newline
 \newline
* The balance brought down could be on either side but must be numerically correct to be awarded. \newline
 \newline
To award the marks amounts must be supported by an appropriate label (recognisable abbreviations are \newline
acceptable for Balance, eg Bal b/d, Bal b/fwd. For Balance b/d accept trial balance shortfall however do \newline
not accept balance c/d as this implies it is carried forward for the next financial period. \newline
Accept a £7 000 credit entry for drawings / wages combined if supported with an appropriate label (ie \newline
wages and drawings (it needs to say both) or journal).   \newline
 \newline
Where a learner makes an entry with the label trade receivables and / or trade payables for £2 400 or £4 \newline
800 this should be ignored and not treated as an alien item.  This could be on either side (or both sides) \newline
of the account. \newline
 \newline
For the OF Balance b/d to be awarded the account must be balanced and arithmetically correct.  \newline
 \newline
 \newline
 \newline
MARK SCHEME – A-LEVEL ACCOUNTING – 7127/1 – JUNE 2022 \newline
12 \newline
Qu \newline
Part \newline
Marking Guidance \newline
Total \newline
marks \newline
13 \newline
2 \newline
Ross had calculated his draft profit to be £86 454 before noticing the errors. \newline
 \newline
Calculate the revised profit figure, taking into account any adjustments required \newline
for correcting the above errors. \newline
4 \newline
 \newline
AO1 – 4 marks \newline
 \newline
  \newline
£ \newline
Draft profit \newline
86 454    \newline
Error 1 \newline
(15 200) (1)  \newline
Error 2 \newline
2 400 \newline
(1) \newline
Error 3 \newline
(3 500) (1) \newline
Revised profit \newline
70 154 \newline
(1) OF * \newline
 \newline
Marker note: \newline
 \newline
Both amount and direction must be correct for each mark to be awarded for items 1 to 3. \newline
 \newline
* The revised profit figure can be awarded as long as the total is arithmetically correct.  Any reference to \newline
trade payables or trade receivables should be ignored and not treated as an alien item. \newline
 \newline
 \newline
 \newline
 \newline
 \newline
MARK SCHEME – A-LEVEL ACCOUNTING – 7127/1 – JUNE 2022 \newline
13 \newline
Section B \newline
 \newline
Qu \newline
Part \newline
Marking Guidance \newline
Total \newline
marks \newline
14 \newline
1 \newline
Prepare a reconciliation of operating profit to net cash flow from operating \newline
activities for the year ended 31 March 2022 to comply with IAS 7. \newline
A full statement of cash flows is not required. \newline
14 \newline
 \newline
AO2 – 14 marks \newline
 \newline
 \newline
£ \newline
Profit from operations \newline
45 667 (5) W1 \newline
Profit on disposal \newline
(4 500) (1) W2 \newline
Depreciation \newline
95 555 (2) OF W3 \newline
Decrease in inventory \newline
2 222 * \newline
Increase in trade receivables \newline
(1 888) * (1) both \newline
Decrease in trade payables \newline
(3 422) (1) \newline
Cash from operations \newline
133 634   \newline
Interest paid \newline
(4 700) ** (1) OF \newline
Tax paid \newline
(5 834) (2) W4 \newline
Net cash flow from operating activities \newline
123 100 (1) OF  \newline
 \newline
W1 Profit from operations: \newline
 \newline
 \newline
£ \newline
Retained earnings c/fwd. \newline
122 030 \newline
Retained earnings b/fwd. \newline
(111 597) \newline
 \newline
10 433 (1) \newline
Taxation \newline
6 534 (1) \newline
Dividends \newline
24 000 (1) \newline
Interest \newline
4 700 (2) \newline
Profit from operations \newline
45 667 OF \newline
 \newline
 \newline
 \newline
 \newline
MARK SCHEME – A-LEVEL ACCOUNTING – 7127/1 – JUNE 2022 \newline
14 \newline
Workings for interest  \newline
£70 000 × 6\% = £4 200 * 7/12 = £2 450 (1) \newline
£90 000 × 6\% = £5 400 * 5/12 = £2 250 (1) \newline
£2 450 + £2 250 = £4 700 \newline
 \newline
Alternative method for calculating the interest \newline
£70,000 x 6\% = £4,200 (1) \newline
£20,000 x 6\% x 5/12 = £500 (1) \newline
£4,200 + £500 = £4,700 \newline
 \newline
 \newline
  \newline
W2 Profit on disposal: \newline
£39 500 – £35 000 = £4 500 (1) \newline
  \newline
 \newline
W3 Depreciation: \newline
Provision For Depreciation \newline
Disposal \newline
27 000 (1) Balance b/d \newline
339 930* (1) \newline
Balance c/d \newline
408 485* Depreciation charge \newline
95 555 OF \newline
 \newline
435 485  \newline
435 485 \newline
 \newline
Marker note:  \newline
 \newline
*Mark is for both items  \newline
 \newline
Alternative: \newline
£408 485 – £339 930 = £68 555 (1) + £27 000 \# (1) = £95 555 OF \newline
\# Student may show add £62 000 and deduct £35 000 so please reward this instead if direction is \newline
correct. \newline
 \newline
Award one mark for the following depreciation figures shown in the reconciliation statement (with or \newline
without workings).   \newline
£68 555 \newline
£27 000 \newline
£41 555 \newline
 \newline
All other figures must be checked via the workings \newline
 \newline
W4 Tax paid: \newline
Tax \newline
Bank \newline
5 834 OF  Balance b/d \newline
1 800* (1) \newline
Balance c/d \newline
2 500 (1) Income statement \newline
6 534* \newline
 \newline
8 334  \newline
8 334 \newline
 \newline
Marker note:  \newline
 \newline
*Mark is for both items  \newline
 \newline
Alternative: \newline
£6 534 + £1 800 = £8 334 (1) – £2 500 (1) = £5 834 OF \newline
 \newline
 \newline
 \newline
MARK SCHEME – A-LEVEL ACCOUNTING – 7127/1 – JUNE 2022 \newline
15 \newline
Marker notes: \newline
 \newline
Calculations where the workings are not transferred or are transferred but in the wrong direction into the \newline
reconciliation will not be fully rewarded. For example a depreciation charge of £95 555 would be \newline
awarded 1 mark if it is not transferred into the reconciliation or if £95 555 is shown as a negative figure in \newline
the reconciliation. A profit on disposal figure of £4 500 would be awarded zero marks if it is not \newline
transferred into the reconciliation or if shown as an addition in the reconciliation. \newline
 \newline
Each entry must have a reasonable narrative (does not have to say increase or decrease as long as the \newline
direction is clear).   \newline
 \newline
** Interest paid must agree with their original figure in their profit from operations figure.  \newline
 \newline
 \newline
 \newline
 \newline
MARK SCHEME – A-LEVEL ACCOUNTING – 7127/1 – JUNE 2022 \newline
16 \newline
Qu \newline
Part \newline
Marking Guidance \newline
Total \newline
marks \newline
14 \newline
2 \newline
Assess the Managing Director’s opinion.  \newline
6 \newline
 \newline
AO2 – 2 marks, AO3 – 4 marks \newline
 \newline
Level \newline
Marks \newline
Description \newline
3 \newline
5–6 \newline
Judgements are fully supported by a wide range of evidence.  A clear and balanced \newline
analysis of data/information/issues is provided, showing a logical chain of reasoning. \newline
2 \newline
3–4 \newline
Judgements are partially supported by evidence.  A reasoned, but unbalanced \newline
analysis of data/information/issues is provided; starts to develop a chain of \newline
reasoning.  Comprehensive and relevant knowledge and understanding of \newline
principles/concepts/techniques has been applied in context. \newline
1 \newline
1–2 \newline
Judgements may be asserted but are unsupported by evidence.  An analysis of \newline
discrete points of data/information/issues provided; no chain of reasoning is \newline
attempted.  Limited but relevant knowledge and understanding of \newline
principles/concepts/techniques has been applied to the context.  \newline
0 \newline
0 \newline
Nothing written worthy of credit.  \newline
 \newline
Answers may include: \newline
 \newline
Case for preparation of a statement of cashflows \newline
 \newline
• Highlights the differences between cash and profit since they are treated differently in the financial \newline
statements so that it is known where cash is generated and spent.  In this instance the business has \newline
generated £123 100 OF from its operating activities, spent £277 500 on non-current assets and \newline
received £127 000 from its financing activities.  This means cash has reduced by £27 400 overall. \newline
• Knowledge of the cash position is important as a business can fail despite being profitable if it does \newline
not have enough cash to meet its obligations. \newline
• Being a public limited company, they should already comply with accounting standards They have to \newline
produce a Statement of Cash Flows to comply with these standards and have their accounts signed \newline
off by the auditors.  If the company’s accounts are not prepared to meet these standards the auditors \newline
will have made comments in the accounts or possibly refused to sign them off. \newline
• Stakeholders can see lots of information which is not shown in the Income Statement and Statement \newline
of Financial Position such as the amount of the dividends paid and changes to the amount of \newline
borrowing more clearly.   This will help stakeholders to make more informed decisions. \newline
• Being on the stock exchange will also involve attracting potential shareholders who may want to see \newline
all of the financial statements. Potential investors want to make an informed decision based on reliable \newline
information that shows a true and fair view.  They would be nervous about investing in the business if \newline
the accounts do not appear to be reliable.  This could result in future shareholders not investing in \newline
Grussell plc and existing shareholders selling their shares which would mean the share price falls and \newline
damages the company’s reputation. \newline
 \newline
 \newline
 \newline
 \newline
 \newline
MARK SCHEME – A-LEVEL ACCOUNTING – 7127/1 – JUNE 2022 \newline
17 \newline
Case against preparation of a statement of cashflows \newline
 \newline
• For someone with limited accounting knowledge the Statement of Cash Flows can seem confusing \newline
and hard to understand compared to an Income Statement and Statement of Financial Position. \newline
• The cost of preparing the accounts and having them audited reduces profitability. \newline
• A lack of understanding of the content of a statement could actually put some investors off investing in \newline
shares which might stop the share price rising as much as it could potentially have done. \newline
• The amount of time and / or money spent on the accounts could have been better spent on improving \newline
the business in other ways. \newline
• Accounts can be window dressed to some degree to make the business look better / worse than it \newline
actually is. \newline
 \newline
Marker notes: \newline
 \newline
Not all content needs to be covered to gain full marks. \newline
 \newline
The indicative content is not exhaustive other credit worthy material should be awarded marks as \newline
appropriate. \newline
 \newline
 \newline
MARK SCHEME – A-LEVEL ACCOUNTING – 7127/1 – JUNE 2022 \newline
18 \newline
Qu \newline
Part \newline
Marking Guidance \newline
Total \newline
marks \newline
15 \newline
1 \newline
Prepare an income statement for Cluedo Coffee for the year ended 31 March \newline
2022. \newline
14 \newline
 \newline
AO2 – 14 marks \newline
 \newline
Cluedo Coffee \newline
Income statement for the year ended 31 March 2022  \newline
 \newline
   £ \newline
 \newline
    £ \newline
 \newline
Revenue \newline
    \newline
74 564 (1) CF W1 \newline
  \newline
    \newline
    \newline
Cost of sales \newline
    \newline
    \newline
Opening inventory \newline
16 276 * \newline
    \newline
Purchases \newline
43 451 ** \newline
    \newline
Less returns outwards \newline
4 200 ** (1) both \newline
    \newline
Less closing inventory \newline
12 304 * (1) both \newline
    \newline
 \newline
    \newline
43 223   \newline
Gross profit \newline
    \newline
31 341 (1) OF \newline
Add other income \newline
    \newline
    \newline
Profit on disposal of vehicle \newline
    \newline
6 000 (1) \newline
Rent receivable \newline
    \newline
3 100 ***  \newline
  \newline
    \newline
40 441   \newline
Less expenses \newline
    \newline
    \newline
General expenses \newline
15 043   \newline
    \newline
Rent paid \newline
9 600 *** (1) both \newline
    \newline
Depreciation of vehicles \newline
5 000 (5) W2 \newline
    \newline
Depreciation of premises \newline
2 400 (1) W3 \newline
    \newline
Loan interest \newline
4 000 (1) W4 \newline
    \newline
  \newline
    \newline
36 043   \newline
Profit for the year \newline
  \newline
  \newline
4 398 (1) OF  \newline
 \newline
 \newline
 \newline
MARK SCHEME – A-LEVEL ACCOUNTING – 7127/1 – JUNE 2022 \newline
19 \newline
Marker notes: \newline
 \newline
Gross profit must be labelled and mathematically correct based on the students own revenue and cost of \newline
sales figures. \newline
 \newline
Profit for the year must include general expenses, be mathematically correct and contain no extraneous \newline
items eg drawings in the expenses. \newline
 \newline
Accept a net purchases figure of £39 251 (1) where a student has not shown purchases and returns \newline
outwards separately. \newline
 \newline
Accept net profit for the label. Do not accept abbreviations eg PFTY or just ‘profit’. \newline
 \newline
W1 Revenue \newline
 \newline
£ \newline
Original revenue \newline
102 564 \newline
less sale of vehicle \newline
28 000 \newline
Revised revenue figure \newline
74 564 \newline
 \newline
 \newline
W2 Depreciation of motor vehicles \newline
 \newline
Cost \newline
£ \newline
Vehicles at cost \newline
83 000 \newline
Less disposal at cost \newline
49 500 \newline
New vehicles at cost \newline
33 500 \newline
 \newline
Depreciation \newline
£ \newline
Provision at 1 April 2021 \newline
46 000  \newline
Disposal depreciation (49 500 – 22 000) \newline
27 500 (1) \newline
Provision prior to depreciation charge for the year \newline
18 500  \newline
 \newline
 \newline
New cost \newline
£33 500  (1) \newline
less new dep \newline
£18 500  (1) \newline
Net book value at 31 March \newline
£15 000  (1) OF \newline
Depreciation rate \newline
33 1/3\%  \newline
Depreciation charge for the year \newline
£5 000  (1) OF \newline
 \newline
Alternative \newline
Cost \newline
 \newline
£83 000  \newline
 \newline
 \newline
less provision for depn at 1 April 2021 \newline
less disposal \newline
£46 000 \newline
£22 000  \newline
 \newline
 \newline
Net book value at 31 March \newline
£15 000  (4) \newline
Depreciation rate \newline
33 1/3\%  \newline
Depreciation charge for the year \newline
£5 000  (1) OF \newline
 \newline
If the working is shown but the answer is not used in the income statement award up to 4 marks \newline
 \newline
W3 Depreciation of property \newline
Property at cost \newline
£120 000 \newline
Depreciation rate \newline
2\% \newline
Depreciation charge for the year \newline
£2 400 \newline
MARK SCHEME – A-LEVEL ACCOUNTING – 7127/1 – JUNE 2022 \newline
20 \newline
 \newline
If an answer of £7 400 (ie combining both depreciation of £5 000 and £2 400) and is shown with a \newline
reasonable label (eg depreciation of non-current assets, depreciation) this can be awarded all 6 marks. \newline
 \newline
  \newline
W4 Loan interest \newline
Loan \newline
£50 000 \newline
Annual interest rate \newline
8\% \newline
Loan interest \newline
£4 000 \newline
 \newline
 \newline
 \newline
MARK SCHEME – A-LEVEL ACCOUNTING – 7127/1 – JUNE 2022 \newline
21 \newline
Qu \newline
Part \newline
Marking Guidance \newline
Total \newline
marks \newline
15 \newline
2 \newline
Assess whether Paulo is correct in his assumption about the profitability of \newline
Cluedo Coffee. \newline
6 \newline
 \newline
AO2 – 2 marks, AO3 4 marks \newline
 \newline
Level \newline
Marks \newline
Description \newline
3 \newline
5–6 \newline
Judgements are fully supported by a wide range of evidence.  A clear and balanced \newline
analysis of data/information/issues is provided, showing a logical chain of reasoning. \newline
2 \newline
3–4 \newline
Judgements are partially supported by evidence.  A reasoned, but unbalanced \newline
analysis of data/information/issues is provided; starts to develop a chain of \newline
reasoning.  Comprehensive and relevant knowledge and understanding of \newline
principles/concepts/techniques has been applied in context. \newline
1 \newline
1–2 \newline
Judgements may be asserted but are unsupported by evidence.  An analysis of \newline
discrete points of data/information/issues provided; no chain of reasoning is \newline
attempted.  Limited but relevant knowledge and understanding of \newline
principles/concepts/techniques has been applied to the context.  \newline
0 \newline
0 \newline
Nothing written worthy of credit.  \newline
 \newline
Answers may include: \newline
 \newline
• Gross profit margin is 42\% (OF) and the profit in relation to revenue is 5.9\% (OF).  Better control of \newline
expenses is needed based on this information. \newline
• A loss (OF) would have been made were it not for the profit of £6 000 on the disposal of a non-current \newline
asset. \newline
• A significant proportion of the profit came from rental income (rent receivable) of £3 100.  If this were \newline
to fall in the future it could cause losses to occur. \newline
• Drawings are significantly higher than the profit made (OF).  This will not help Cluedo Coffee in the \newline
future as there is less money available to use for any expansion. \newline
• Although revenue has grown by 10\% profitability is very low and we do not know if profit grew by a \newline
similar amount. \newline
• Loan interest of £4 000 will significantly reduce profit until it is repaid. \newline
• It would be helpful to have ratios from similar business and/or last years figures to make a more \newline
informed decision on its profitability trend. \newline
 \newline
 \newline
 \newline
MARK SCHEME – A-LEVEL ACCOUNTING – 7127/1 – JUNE 2022 \newline
22 \newline
Section C \newline
 \newline
Qu \newline
Part \newline
Marking Guidance \newline
Total \newline
marks \newline
16 \newline
 \newline
Evaluate which of the two options Otmar should choose.  Justify your answer. \newline
 \newline
Consider both financial and non-financial factors. \newline
25 \newline
 \newline
AO2 – 5 marks, AO3 – 20 marks \newline
 \newline
Level  \newline
 Marks  Description \newline
5 \newline
21–25 \newline
A clear and balanced response that presents a coherent and logically reasoned \newline
judgement and conclusion/solution that is supported by an astute consideration of a \newline
wide range of evidence including other factors relevant to the wider context.  \newline
 \newline
There is an insightful assessment of the significance and limitations of the evidence \newline
used to support the judgement. \newline
4 \newline
16–20 \newline
A reasoned, but in places unbalanced, judgement and conclusion/solution is \newline
presented that is supported by an evaluation of a wide range of evidence, including a \newline
narrow consideration of other factors relevant to the wider context.  \newline
 \newline
There is a partial assessment of the significance and limitations of the evidence used \newline
to support the judgement. \newline
3 \newline
11–15 \newline
An underdeveloped judgement and conclusion/solution is presented that is \newline
supported by an evaluation of a range of evidence provided in the question; however \newline
there may be inconsistencies and the reasoning may contain inaccuracies. \newline
 \newline
A comprehensive and relevant selection of information is analysed, showing a \newline
developed logical chain of reasoning.  The results of any appropriate calculation/s \newline
are integrated into the analysis and evaluations offered on most.  \newline
 \newline
Comprehensive and relevant knowledge and understanding of \newline
principles/concepts/techniques is drawn together and applied successfully to the \newline
context.  Where appropriate, a thorough selection of relevant calculations is \newline
attempted; these may include minor errors. \newline
2 \newline
6–10 \newline
A basic judgement and conclusion/solution is presented, it is supported by a limited \newline
evaluation of evidence provided in the question, containing significant inaccuracies.  \newline
 \newline
A limited but relevant selection of information is analysed, starting to develop a \newline
logical chain of reasoning.  The results of the calculation/s are integrated into the \newline
analysis but with weak evaluations.  \newline
 \newline
Limited but relevant knowledge and understanding of principles/concepts/techniques \newline
is drawn together and applied successfully to the context.  Where appropriate, a \newline
limited selection of relevant calculations is attempted; these may include minor \newline
errors. \newline
 \newline
 \newline
 \newline
MARK SCHEME – A-LEVEL ACCOUNTING – 7127/1 – JUNE 2022 \newline
23 \newline
1 \newline
1–5 \newline
A judgement and conclusion/solution may be asserted, but it is unsupported by any \newline
evidence. \newline
 \newline
Responses present a limited selection of information that is not wholly relevant with \newline
an attempt at analysis.  A chain of reasoning ranges from being barely present to \newline
undeveloped.  \newline
 \newline
Fragmented items of knowledge and understanding of \newline
principles/concepts/techniques relevant to the contexts are present.  These are likely \newline
to be descriptive, with limited application to the context.  Where appropriate, some \newline
calculations are attempted; these are likely to contain errors and may not be relevant \newline
to the context.  Results of the calculations are stated with little or no evaluation. \newline
0 \newline
0 \newline
Nothing written worthy of credit. \newline
 \newline
Indicative content may include: \newline
 \newline
General comments \newline
 \newline
Application AO2 \newline
• Purchasing the machinery is needed to ensure further orders can be completed – potential for an \newline
increase of 25\% in output which could improve profit by £875 per month (£10 500 per annum) to £4 \newline
375 per month (£52 500 per annum). \newline
 \newline
  \newline
Making Kimmi a partner \newline
 \newline
Application AO2 \newline
• Slight reduction in Kimmi’s salary by £6 000 which reduces the risk to Otmar and gives him a higher \newline
profit figure before distribution. \newline
• Current annual profits are £42 000 which all belong to Otmar.  If he takes up the partnership it will be \newline
£42 000 + £10 500 + £31 000 (no salary to pay Kimmi) = £83 500.  After deducting the salaries          \newline
of £53 500 there could be £30 000 available which would be split 2:1. \newline
 \newline
Case for choosing Kimmi  \newline
 \newline
Analysis and evaluation AO3 \newline
• Otmar could receive £48 500 which is an improvement on the £42 000 by £6 500. \newline
• Kimmi will receive £35 000 (£25 000 salary + £10 000 profit share) which means Kimmi is better off by \newline
£4 000. \newline
• Risk of losses are now shared which spreads risk for Otmar. \newline
• Making Kimmi a partner could help avoid Kimmi leaving and any potential loss of staff that could \newline
follow.  This helps keep the skills and experience in the business. \newline
• Ideas could increase if Kimmi is now more motivated to try and boost profitability. \newline
• Add in monthly figures  \newline
  \newline
Case against choosing Kimmi \newline
 \newline
Analysis and Evaluation AO3 \newline
• Profits will now need to be shared. \newline
• Decisions will need to be discussed with Kimmi. \newline
• Kimmi is not putting as much money in so Otmar needs to make a further injection of his own capital. \newline
MARK SCHEME – A-LEVEL ACCOUNTING – 7127/1 – JUNE 2022 \newline
24 \newline
• Otmar is receiving less investment than with Priya although Kimmi will only own 1/3 of the business. \newline
• Conflict could escalate further now that Kimmi has more power and influence as an owner and her \newline
strong influence over the staff. \newline
• Does Kimmi have enough management experience to successfully part own a business? \newline
 \newline
Making Priya a partner \newline
 \newline
General comments \newline
 \newline
Application AO2 \newline
• No formal partnership agreement means Otmar will not receive a salary for working there \newline
• Partners will share profit equally despite Priya not being involved in the running of the business \newline
• Current annual profits are £42 000 (3500 × 12) which all belong to Otmar.  If he takes up the \newline
partnership if will be £42 000 + £10 500 = £52 500.  Profits are shared equally so Otmar would only \newline
receive £26 250 as his share of the profits. \newline
  \newline
Case for choosing Priya \newline
 \newline
Analysis and Evaluation AO3 \newline
• No change to the way the business runs if Priya stays as a silent partner so the business could still be \newline
run the way Otmar wishes \newline
• Priya is taking a 50\% stake but is putting in proportionally much more than this.  The amount could be \newline
enough to repay any non-current liabilities / bank overdrafts and / or buy the machinery.  It could also \newline
be used to boost profits from interest saved and reduce the issues they have with cash flow which \newline
might be beneficial if the additional orders are on the usual credit terms \newline
• Priya has experience in the industry and could provide knowledge and industry contracts which could \newline
further boost profitability. \newline
  \newline
Case against choosing Priya \newline
 \newline
• Could upset Kimmi as she does not achieve her partner ambition and could leave to set up her own \newline
business (possibly taking staff and customer contracts with her potentially) \newline
• Priya may not remain a silent partner \newline
• Otmar will be £15 750 worse off (£42 000 – £26 250). \newline
 \newline
Limitations of the data \newline
• Sales increase of 25\% – how has this been estimated and is it realistic?  Will they be able to sell the \newline
additional units they produce? \newline
• Very limited financial data is provided such as sales revenue / profits in recent times which could have \newline
helped with a decision \newline
• How would the changing role of Kimmi affect their working relationship? \newline
• Could Otmar work with Priya?  Would it affect their friendship? \newline
• Are there other staff who might wish to become owners in other areas of the business? \newline
 \newline
Marker note: \newline
 \newline
The indicative content is not exhaustive: other creditworthy material should be awarded marks as \newline
appropriate. \newline
 \newline
 \newline
 \newline
MARK SCHEME – A-LEVEL ACCOUNTING – 7127/1 – JUNE 2022 \newline
25 \newline
Qu \newline
Part \newline
Marking Guidance \newline
Total \newline
marks \newline
17 \newline
 \newline
Evaluate both businesses from Donna’s perspective as a potential investor.  \newline
Make a recommendation on how she should invest her savings. \newline
 \newline
Consider both financial and non-financial factors. \newline
25 \newline
 \newline
AO2 – 5 marks, AO3 – 20 marks \newline
 \newline
Level \newline
Marks \newline
Description \newline
5 \newline
21–25 \newline
A clear and balanced response that presents a coherent and logically reasoned \newline
judgement and conclusion/solution that is supported by an astute consideration of a \newline
wide range of evidence including other factors relevant to the wider context.  \newline
 \newline
There is an insightful assessment of the significance and limitations of the evidence \newline
used to support the judgement. \newline
4 \newline
16–20 \newline
A reasoned, but in places unbalanced, judgement and conclusion/solution is \newline
presented that is supported by an evaluation of a wide range of evidence, including a \newline
narrow consideration of other factors relevant to the wider context.  \newline
 \newline
There is a partial assessment of the significance and limitations of the evidence used \newline
to support the judgement. \newline
3 \newline
11–15 \newline
An underdeveloped judgement and conclusion/solution is presented that is \newline
supported by an evaluation of a range of evidence provided in the question; however \newline
there may be inconsistencies and the reasoning may contain inaccuracies. \newline
 \newline
A comprehensive and relevant selection of information is analysed, showing a \newline
developed logical chain of reasoning.  The results of any appropriate calculation/s \newline
are integrated into the analysis and evaluations offered on most.  \newline
 \newline
Comprehensive and relevant knowledge and understanding of \newline
principles/concepts/techniques is drawn together and applied successfully to the \newline
context.  Where appropriate, a thorough selection of relevant calculations is \newline
attempted; these may include minor errors. \newline
2 \newline
6–10 \newline
A basic judgement and conclusion/solution is presented, it is supported by a limited \newline
evaluation of evidence provided in the question, containing significant inaccuracies.  \newline
 \newline
A limited but relevant selection of information is analysed, starting to develop a \newline
logical chain of reasoning.  The results of the calculation/s are integrated into the \newline
analysis but with weak evaluations.  \newline
 \newline
Limited but relevant knowledge and understanding of principles/concepts/techniques \newline
is drawn together and applied successfully to the context.  Where appropriate, a \newline
limited selection of relevant calculations is attempted; these may include minor \newline
errors. \newline
 \newline
 \newline
 \newline
MARK SCHEME – A-LEVEL ACCOUNTING – 7127/1 – JUNE 2022 \newline
26 \newline
1 \newline
1–5 \newline
A judgement and conclusion/solution may be asserted, but it is unsupported by any \newline
evidence. \newline
 \newline
Responses present a limited selection of information that is not wholly relevant with \newline
an attempt at analysis.  A chain of reasoning ranges from being barely present to \newline
undeveloped.  \newline
 \newline
Fragmented items of knowledge and understanding of \newline
principles/concepts/techniques relevant to the contexts are present.  These are likely \newline
to be descriptive, with limited application to the context.  Where appropriate, some \newline
calculations are attempted; these are likely to contain errors and may not be relevant \newline
to the context.  Results of the calculations are stated with little or no evaluation \newline
0 \newline
0 \newline
Nothing written worthy of credit. \newline
 \newline
Answers may include: \newline
 \newline
Case for purchasing shares in Pik plc \newline
 \newline
AO2 Application \newline
• Can purchase 25 000 shares (£50 000 / £2.00). \newline
• Dividend could be (based on 2022) (50 000 × 2.5\%) = £1 250 per annum. \newline
• Earnings per share is fluctuating. \newline
• Dividend cover appears to be declining. \newline
• Market price appears to fluctuate although the overall trend is an improvement of 60p. \newline
• Pik plc has very little debt so more available for shareholders.  Barnard plc has mortgages on a \newline
number of properties so has an increased risk for investors and the mortgage interest must be paid \newline
before dividends. \newline
• Pik plc has an online business model which may keep costs lower. \newline
• Ethically, Pik plc might face consumer backlash if the users are aware of the rumours whether they \newline
are true or not. \newline
• Donna could be left with expensive shares if consumers switch to a new social media app / site as \newline
many app’s tend to have a short product life. \newline
 \newline
AO3 Analysis and evaluation \newline
• A yearly total dividend of £1 250 is significantly higher than Barnard plc (£800) by £450 so regular \newline
dividend earnings could be possible each year if this is what Donna is looking for.  However an \newline
increase in share price might be equally desirable \newline
• Pik plc is predominantly an online only company so it may be able to keep costs lower which boost \newline
profitability and thus dividends in the longer term \newline
• Arguably greater risk overall than Barnard plc although the dividend yield also provides a greater \newline
reward. \newline
• If found guilty of breaching GDPR / data protection fines could wipe out / reduce their profits \newline
considerably  \newline
• Earnings per share is very high but fluctuates so hard to ascertain the true performance trend – what \newline
direction will it go in next? \newline
• Dividend cover is falling and is lower than Barnard plc.  This could suggest a greater return to \newline
shareholders which may be good for shareholders in the very short term but could result in a lack of \newline
investment in the longer term   \newline
• The recent (2022) fall in earning per share suggests profitability is falling unless a share issue has \newline
occurred.   \newline
• Dividend yield has fluctuated as a result of the fluctuation in share price and dividends.  It is unclear to \newline
see which direction this is likely to go next. \newline
MARK SCHEME – A-LEVEL ACCOUNTING – 7127/1 – JUNE 2022 \newline
27 \newline
Case for Barnard plc \newline
 \newline
AO2 Application \newline
• Can purchase 35 714 shares (50 000 / £1.40) which is a bigger quantity of shares than could be \newline
purchased in Pik plc. \newline
• Dividends could be (based on 2022) 50 000 × 1.6\% = £800 per annum which is less than Pik plc \newline
despite Donna owning more shares. \newline
• Earnings per share is high and growing. \newline
• Dividend cover is growing – suggesting lots of profit is being retained. \newline
• Market price has more than doubled suggesting lots of investor confidence in the company’s future. \newline
• Established name in an established market for investors wishing for more security. \newline
• Land ownership provides security for the business in the future. \newline
• Good ethical stance which might be appreciated by its shareholders, potential shareholders and \newline
customers.  The company might be less likely to suffer negative publicity which could harm its share \newline
price. \newline
 \newline
AO3 Analysis and evaluation \newline
• Dividends could be (based on 2022) 50 000 × 1.6\% = £800 per annum which is less than Pik plc \newline
despite Donna owning more shares. \newline
• Dividend cover is high and growing each year.  This would indicate investment through retained \newline
earnings which may be good for investors in the longer term although possibly less so in the  \newline
short term. \newline
• Earnings per share is very high but fluctuates – what direction will it go in next? \newline
• Dividend yield has stayed broadly the same so the increase in share price must mean an increase in \newline
the dividend per share. \newline
• Rising market price suggests stock market confidence in the company which may make it a more \newline
attractive deal in the longer term.  However she will own fewer shares than if she had bought the \newline
shares before the share price increases. \newline
• Barnard plc has lots of land of significant value which could be sold to generate cash to pay future \newline
dividends. \newline
• Barnard plc has a strong ethical stance in the way it runs its business.  The company might be less \newline
likely to suffer negative publicity impacting on its share price as a result.  This could mean Donna’s \newline
investment is safer. \newline
 \newline
Limitations could include \newline
• Ratios are all historic data – what is their current trading position?  This data is at least 2 months out of \newline
date. \newline
• Current market price of shares is unknown so the decision to buy shares could be affected by some \newline
more recent share price changes.  Depending where the current price is this could have a significant \newline
impact on the attractiveness of buying the shares as the figures might need recalculating. \newline
• More data could help the decision.  For example we are not told the interest cover.  Pik plc has low \newline
debt so has less risk attached to repayment of debt. \newline
• Ratios do not provide the actual figures so could be masking a bigger problem that could be seen by \newline
looking at the accounts as a whole.  The financial statements could have been window dressed.   For \newline
example the data implies that Pik plc is lowly geared but this ratio is not given. \newline
• Ratios do not take into consideration other non-financial factors such as the company changing its \newline
strategic direction, management experience, investment plans, sources of their finance. \newline
• Market conditions of the two companies may be very different since they are in different sectors of \newline
business.  It is difficult to make a true comparison between them as a result. \newline
• The year 2020 could just be an anomaly in terms of the share price for both companies.  What was the \newline
share price of each company before?  Would be better to have a five-year history. \newline
 \newline
 \newline
MARK SCHEME – A-LEVEL ACCOUNTING – 7127/1 – JUNE 2022 \newline
28 \newline
Evaluation \newline
• There could be other better opportunities to invest elsewhere.  With such a large amount Donna is \newline
likely to want to spread the risk by investing money into both Pik plc and Barnard plc and/or other \newline
companies. \newline
• Non-financial factors could be important to decision making – what is Donna’s ethical stance on \newline
investment.  Would she be prepared to invest in a business who may be using customer data \newline
inappropriately? \newline
• Short-term gain vs potential long-term prospects should be considered.  Barnard plc appears to be \newline
more consistent and less risky.  What is Donna’s investment style – is she prepared to take risks for \newline
greater gain? \newline
 \newline
Marker note: \newline
 \newline
The indicative content is not exhaustive: other creditworthy material should be awarded marks as \newline
appropriate. \newline


    \textbf{Table 1:}
\begin{tabular}{lll}
\toprule
Where you identify: & Situation & Comment to use \\
\midrule
NaN & Application is fragmented or & NaN \\
Application & descriptive or not adequately & Weak application \\
knowledge of & applied to the context & NaN \\
principles/concepts/techniques & Application is relevant and & NaN \\
NaN & applied fully to the context & Clear application \\
NaN & A limited attempt at analysis & Weak analysis \\
Analysis & NaN & NaN \\
NaN & Analysis is logical/considered & Reasoned analysis \\
\bottomrule
\end{tabular}

\textbf{Table 2:}
\begin{tabular}{lll}
\toprule
Evaluation & An attempt at
assessment/evaluation with
little or no supporting
evidence & Weak evaluation \\
\midrule
NaN & Evaluation/assessment is
logical and supported by
evidence & Supported evaluation \\
NaN & Evaluation/assessment
considers the relative
significance and limitations of
the evidence & Astute evaluation \\
Judgement or Conclusion or
Recommendation & An attempt at judgement is
made but unsupported by
evidence or argument & Judgement/conclusion –
unsupported \\
A judgement is made and is
supported but the support is
weak or evidence used is
limited & Judgement/conclusion –
limited/weak support & NaN \\
The judgement is supported
by evidence and argument
but may not be fully balanced & Judgement/conclusion –
supported & NaN \\
Judgement is supported by
evidence and considers the
limitations of the evidence in
context & Judgement/conclusion –
fully justified & NaN \\
\bottomrule
\end{tabular}

\textbf{Table 3:}
\begin{tabular}{rll}
\toprule
Question
Number & Answer & Unnamed: 0 \\
\midrule
1 & D & Neither statement is true. \\
2 & A & Income and expenditure are matched to the period they belong to. \\
3 & D & Current assets− closing inventory

Current liabilities \\
4 & C & Interest on drawings, partner salaries, share of residual profits \\
5 & C & £16 900 \\
6 & C & £1 879 \\
7 & B & £525 Dr \\
8 & C & £37 800 \\
9 & C & £245 \\
10 & B & Partial omission \\
\bottomrule
\end{tabular}

\textbf{Table 4:}
\begin{tabular}{rrlr}
\toprule
Qu & Part & Marking Guidance & Total
marks \\
\midrule
11 & NaN & Explain two advantages of using a bank overdraft as a source of finance for a
business. & 6 \\
\bottomrule
\end{tabular}

\textbf{Table 5:}
\begin{tabular}{rrl}
\toprule
Level & Marks & Description \\
\midrule
3 & 3 & A clear and thorough explanation showing understanding of an advantage of using a
bank overdraft as a source of finance. \\
2 & 2 & A partial explanation showing understanding, but lacking detail and/or minor
inaccuracies. \\
1 & 1 & Fragmented points made. \\
0 & 0 & Nothing worthy of credit. \\
\bottomrule
\end{tabular}

\textbf{Table 6:}
\begin{tabular}{rrlr}
\toprule
Qu & Part & Marking Guidance & Total
marks \\
\midrule
12 & NaN & Prepare the rent receivable account for the year ended 31 March 2022.
Show clearly the amount to be transferred to the income statement and the
balance brought down 1 April 2022. & 5 \\
\bottomrule
\end{tabular}

\textbf{Table 7:}
\begin{tabular}{llll}
\toprule
Details & Amount
£ & Details.1 & Amount
£.1 \\
\midrule
Balance b/d & 860(1) & Bank & 13 900* \\
Bank & 500(1)* both & NaN & NaN \\
Income Statement & 10 720(1)** OF & NaN & NaN \\
Balance c/d & 1 820*** & NaN & NaN \\
NaN & 13 900 & NaN & 13 900 \\
NaN & NaN & Balance b/d & 1 820(2) W1 *** \\
\bottomrule
\end{tabular}

\textbf{Table 8:}
\begin{tabular}{rrlr}
\toprule
Qu & Part & Marking Guidance & Total
marks \\
\midrule
13 & 1 & Prepare Ross’s suspense account to correct the errors, clearly showing the
opening balance. & 5 \\
\bottomrule
\end{tabular}

\textbf{Table 9:}
\begin{tabular}{llll}
\toprule
Details & Amount
£ & Details.1 & Amount
£.1 \\
\midrule
Balance b/d & 19 800(1) OF * & Rent payable & 15 200(1) \\
Discount received & 2 400(1) & Wages & 3 500(1) \\
NaN & NaN & Drawings & 3 500(1) \\
NaN & 22 200 & NaN & 22 200 \\
\bottomrule
\end{tabular}

\textbf{Table 10:}
\begin{tabular}{rrlr}
\toprule
Qu & Part & Marking Guidance & Total
marks \\
\midrule
13 & 2 & Ross had calculated his draft profit to be £86 454 before noticing the errors.

Calculate the revised profit figure, taking into account any adjustments required
for correcting the above errors. & 4 \\
\bottomrule
\end{tabular}

\textbf{Table 11:}
\begin{tabular}{ll}
\toprule
Unnamed: 0 & £ \\
\midrule
Draft profit & 86 454 \\
Error 1 & (15 200)(1) \\
Error 2 & 2 400(1) \\
Error 3 & (3 500)(1) \\
Revised profit & 70 154(1) OF * \\
\bottomrule
\end{tabular}

\textbf{Table 12:}
\begin{tabular}{rrlr}
\toprule
Qu & Part & Marking Guidance & Total
marks \\
\midrule
14 & 1 & Prepare a reconciliation of operating profit to net cash flow from operating
activities for the year ended 31 March 2022 to comply with IAS 7.
A full statement of cash flows is not required. & 14 \\
\bottomrule
\end{tabular}

\textbf{Table 13:}
\begin{tabular}{ll}
\toprule
Unnamed: 0 & £ \\
\midrule
Profit from operations & 45 667(5) W1 \\
Profit on disposal & (4 500)(1) W2 \\
Depreciation & 95 555(2) OF W3 \\
Decrease in inventory & 2 222* \\
Increase in trade receivables & (1 888)* (1) both \\
Decrease in trade payables & (3 422)(1) \\
Cash from operations & 133 634 \\
Interest paid & (4 700)** (1) OF \\
Tax paid & (5 834)(2) W4 \\
Net cash flow from operating activities & 123 100(1) OF \\
\bottomrule
\end{tabular}

\textbf{Table 14:}
\begin{tabular}{ll}
\toprule
Unnamed: 0 & £ \\
\midrule
Retained earnings c/fwd. & 122 030 \\
Retained earnings b/fwd. & (111 597) \\
NaN & 10 433 (1) \\
Taxation & 6 534 (1) \\
Dividends & 24 000 (1) \\
Interest & 4 700 (2) \\
Profit from operations & 45 667 OF \\
\bottomrule
\end{tabular}

\textbf{Table 15:}
\begin{tabular}{llll}
\toprule
Disposal & 27 000 (1) & Balance b/d & 339 930* (1) \\
\midrule
Balance c/d & 408 485* & Depreciation charge & 95 555 OF \\
NaN & 435 485 & NaN & 435 485 \\
\bottomrule
\end{tabular}

\textbf{Table 16:}
\begin{tabular}{llll}
\toprule
Bank & 5 834 OF & Balance b/d & 1 800* (1) \\
\midrule
Balance c/d & 2 500 (1) & Income statement & 6 534* \\
NaN & 8 334 & NaN & 8 334 \\
\bottomrule
\end{tabular}

\textbf{Table 17:}
\begin{tabular}{rrlr}
\toprule
Qu & Part & Marking Guidance & Total
marks \\
\midrule
14 & 2 & Assess the Managing Director’s opinion. & 6 \\
\bottomrule
\end{tabular}

\textbf{Table 18:}
\begin{tabular}{rll}
\toprule
Level & Marks & Description \\
\midrule
3 & 5–6 & Judgements are fully supported by a wide range of evidence.  A clear and balanced
analysis of data/information/issues is provided, showing a logical chain of reasoning. \\
2 & 3–4 & Judgements are partially supported by evidence.  A reasoned, but unbalanced
analysis of data/information/issues is provided; starts to develop a chain of
reasoning.  Comprehensive and relevant knowledge and understanding of
principles/concepts/techniques has been applied in context. \\
1 & 1–2 & Judgements may be asserted but are unsupported by evidence.  An analysis of
discrete points of data/information/issues provided; no chain of reasoning is
attempted.  Limited but relevant knowledge and understanding of
principles/concepts/techniques has been applied to the context. \\
0 & 0 & Nothing written worthy of credit. \\
\bottomrule
\end{tabular}

\textbf{Table 19:}
\begin{tabular}{rrlr}
\toprule
Qu & Part & Marking Guidance & Total
marks \\
\midrule
15 & 1 & Prepare an income statement for Cluedo Coffee for the year ended 31 March
2022. & 14 \\
\bottomrule
\end{tabular}

\textbf{Table 20:}
\begin{tabular}{lll}
\toprule
Unnamed: 0 & £ & £.1 \\
\midrule
Revenue & NaN & 74 564(1) CF W1 \\
NaN & NaN & NaN \\
Cost of sales & NaN & NaN \\
Opening inventory & 16 276* & NaN \\
Purchases & 43 451** & NaN \\
Less returns outwards & 4 200** (1) both & NaN \\
Less closing inventory & 12 304* (1) both & NaN \\
NaN & NaN & 43 223 \\
Gross profit & NaN & 31 341(1) OF \\
Add other income & NaN & NaN \\
Profit on disposal of vehicle & NaN & 6 000(1) \\
Rent receivable & NaN & 3 100*** \\
NaN & NaN & 40 441 \\
Less expenses & NaN & NaN \\
General expenses & 15 043 & NaN \\
Rent paid & 9 600*** (1) both & NaN \\
Depreciation of vehicles & 5 000(5) W2 & NaN \\
Depreciation of premises & 2 400(1) W3 & NaN \\
Loan interest & 4 000(1) W4 & NaN \\
NaN & NaN & 36 043 \\
Profit for the year & NaN & 4 398(1) OF \\
\bottomrule
\end{tabular}

\textbf{Table 21:}
\begin{tabular}{rrlr}
\toprule
Qu & Part & Marking Guidance & Total
marks \\
\midrule
15 & 2 & Assess whether Paulo is correct in his assumption about the profitability of
Cluedo Coffee. & 6 \\
\bottomrule
\end{tabular}

\textbf{Table 22:}
\begin{tabular}{rll}
\toprule
Level & Marks & Description \\
\midrule
3 & 5–6 & Judgements are fully supported by a wide range of evidence.  A clear and balanced
analysis of data/information/issues is provided, showing a logical chain of reasoning. \\
2 & 3–4 & Judgements are partially supported by evidence.  A reasoned, but unbalanced
analysis of data/information/issues is provided; starts to develop a chain of
reasoning.  Comprehensive and relevant knowledge and understanding of
principles/concepts/techniques has been applied in context. \\
1 & 1–2 & Judgements may be asserted but are unsupported by evidence.  An analysis of
discrete points of data/information/issues provided; no chain of reasoning is
attempted.  Limited but relevant knowledge and understanding of
principles/concepts/techniques has been applied to the context. \\
0 & 0 & Nothing written worthy of credit. \\
\bottomrule
\end{tabular}

\textbf{Table 23:}
\begin{tabular}{rrlr}
\toprule
Qu & Part & Marking Guidance & Total
marks \\
\midrule
16 & NaN & Evaluate which of the two options Otmar should choose.  Justify your answer.

Consider both financial and non-financial factors. & 25 \\
\bottomrule
\end{tabular}

\textbf{Table 24:}
\begin{tabular}{rll}
\toprule
Level & Marks & Description \\
\midrule
5 & 21–25 & A clear and balanced response that presents a coherent and logically reasoned
judgement and conclusion/solution that is supported by an astute consideration of a
wide range of evidence including other factors relevant to the wider context.

There is an insightful assessment of the significance and limitations of the evidence
used to support the judgement. \\
4 & 16–20 & A reasoned, but in places unbalanced, judgement and conclusion/solution is
presented that is supported by an evaluation of a wide range of evidence, including a
narrow consideration of other factors relevant to the wider context.

There is a partial assessment of the significance and limitations of the evidence used
to support the judgement. \\
3 & 11–15 & An underdeveloped judgement and conclusion/solution is presented that is
supported by an evaluation of a range of evidence provided in the question; however
there may be inconsistencies and the reasoning may contain inaccuracies.

A comprehensive and relevant selection of information is analysed, showing a
developed logical chain of reasoning.  The results of any appropriate calculation/s
are integrated into the analysis and evaluations offered on most.

Comprehensive and relevant knowledge and understanding of
principles/concepts/techniques is drawn together and applied successfully to the
context.  Where appropriate, a thorough selection of relevant calculations is
attempted; these may include minor errors. \\
2 & 6–10 & A basic judgement and conclusion/solution is presented, it is supported by a limited
evaluation of evidence provided in the question, containing significant inaccuracies.

A limited but relevant selection of information is analysed, starting to develop a
logical chain of reasoning.  The results of the calculation/s are integrated into the
analysis but with weak evaluations.

Limited but relevant knowledge and understanding of principles/concepts/techniques
is drawn together and applied successfully to the context.  Where appropriate, a
limited selection of relevant calculations is attempted; these may include minor
errors. \\
\bottomrule
\end{tabular}

\textbf{Table 25:}
\begin{tabular}{rll}
\toprule
Unnamed: 0 & Unnamed: 1 & A judgement and conclusion/solution may be asserted, but it is unsupported by any \\
\midrule
NaN & NaN & evidence. \\
NaN & NaN & NaN \\
NaN & NaN & Responses present a limited selection of information that is not wholly relevant with \\
NaN & NaN & an attempt at analysis.  A chain of reasoning ranges from being barely present to \\
1.000000 & 1–5 & undeveloped. \\
NaN & NaN & NaN \\
NaN & NaN & Fragmented items of knowledge and understanding of \\
NaN & NaN & principles/concepts/techniques relevant to the contexts are present.  These are likely \\
NaN & NaN & to be descriptive, with limited application to the context.  Where appropriate, some \\
NaN & NaN & calculations are attempted; these are likely to contain errors and may not be relevant \\
NaN & NaN & to the context.  Results of the calculations are stated with little or no evaluation. \\
0.000000 & 0 & Nothing written worthy of credit. \\
\bottomrule
\end{tabular}

\textbf{Table 26:}
\begin{tabular}{rrlr}
\toprule
Qu & Part & Marking Guidance & Total
marks \\
\midrule
17 & NaN & Evaluate both businesses from Donna’s perspective as a potential investor.
Make a recommendation on how she should invest her savings.

Consider both financial and non-financial factors. & 25 \\
\bottomrule
\end{tabular}

\textbf{Table 27:}
\begin{tabular}{rll}
\toprule
Level & Marks & Description \\
\midrule
5 & 21–25 & A clear and balanced response that presents a coherent and logically reasoned
judgement and conclusion/solution that is supported by an astute consideration of a
wide range of evidence including other factors relevant to the wider context.

There is an insightful assessment of the significance and limitations of the evidence
used to support the judgement. \\
4 & 16–20 & A reasoned, but in places unbalanced, judgement and conclusion/solution is
presented that is supported by an evaluation of a wide range of evidence, including a
narrow consideration of other factors relevant to the wider context.

There is a partial assessment of the significance and limitations of the evidence used
to support the judgement. \\
3 & 11–15 & An underdeveloped judgement and conclusion/solution is presented that is
supported by an evaluation of a range of evidence provided in the question; however
there may be inconsistencies and the reasoning may contain inaccuracies.

A comprehensive and relevant selection of information is analysed, showing a
developed logical chain of reasoning.  The results of any appropriate calculation/s
are integrated into the analysis and evaluations offered on most.

Comprehensive and relevant knowledge and understanding of
principles/concepts/techniques is drawn together and applied successfully to the
context.  Where appropriate, a thorough selection of relevant calculations is
attempted; these may include minor errors. \\
2 & 6–10 & A basic judgement and conclusion/solution is presented, it is supported by a limited
evaluation of evidence provided in the question, containing significant inaccuracies.

A limited but relevant selection of information is analysed, starting to develop a
logical chain of reasoning.  The results of the calculation/s are integrated into the
analysis but with weak evaluations.

Limited but relevant knowledge and understanding of principles/concepts/techniques
is drawn together and applied successfully to the context.  Where appropriate, a
limited selection of relevant calculations is attempted; these may include minor
errors. \\
\bottomrule
\end{tabular}

\textbf{Table 28:}
\begin{tabular}{rll}
\toprule
Unnamed: 0 & Unnamed: 1 & A judgement and conclusion/solution may be asserted, but it is unsupported by any \\
\midrule
NaN & NaN & evidence. \\
NaN & NaN & NaN \\
NaN & NaN & Responses present a limited selection of information that is not wholly relevant with \\
NaN & NaN & an attempt at analysis.  A chain of reasoning ranges from being barely present to \\
1.000000 & 1–5 & undeveloped. \\
NaN & NaN & Fragmented items of knowledge and understanding of \\
NaN & NaN & principles/concepts/techniques relevant to the contexts are present.  These are likely \\
NaN & NaN & to be descriptive, with limited application to the context.  Where appropriate, some \\
NaN & NaN & calculations are attempted; these are likely to contain errors and may not be relevant \\
NaN & NaN & to the context.  Results of the calculations are stated with little or no evaluation \\
0.000000 & 0 & Nothing written worthy of credit. \\
\bottomrule
\end{tabular}


    \end{document}
    